\documentclass{article}

\usepackage[margin=1in]{geometry}

\title{2021 Castro Valley Junior Math Tournament \\ Elementary School Division Solutions}
\author{}
\date{}

\begin{document}
\maketitle

\section{Moocraft}
$40$ minutes is $40 \cdot 60 = 2400$ seconds.
Three-fourths of this is $2400 \cdot \frac{3}{4} = 1800$.
Five-sevenths of $1800$ is $1800 \cdot \frac{5}{7} \approx 1285.71$.
Therefore the answer rounded to the nearest integer is $1286$ seconds.

\section{Farmer Pearson's Pears}
Five dozen is $5 \cdot 12 = 60$.
$60$ pears is $\frac{60}{18} \approx 3.33$ packs.
Since Farmer Pearson only sells whole packs, Cowboy Alex needs to buy $4$ packs.

\section{Cowbe}
After Bessie chose her corner, there are seven corners left for Elsie to choose.
Three of those corners are connected with Bessie's corner by an edge.
Therefore, the probability is $\frac{3}{7}$.

\section{Moolan}
If Moolan's age is $x$, then Bessie's age is $3x - 50$.
Quadrupling Bessie's age and subtracting $53$ results in Moolan's age, so $4(3x - 50) - 53 = x$.
Expanding the product on the left side using the distributitve property, we get $12x - 200 - 53 = x$.
This means that $11x = 253$ so $x = 23$.
Therefore, Moolan is $23$ years old.

\section{Accowntant}
Billy has $\$21$ more than Bobby, Bailey has $\$13$ more than Billy, and Betty has $\$17$ more than Bailey.
Therefore, Betty has $21 + 13 + 17 = 51$ dollars more than Bobby.

\section{Moovie Night}
$10$ out of $18$ grams of food or drink that Bessie consumes is popcorn, so the amount of popcorn that she consumed is $\frac{10}{18} \cdot 3060 = 1700$ grams.
Alternatively, we can use a system of equations.
If $a$ is the amount of popcorn that Bessie consumed in grams and $b$ is the amount of Cocow-Cola that she consumed in grams, we have $\frac{a}{10} = \frac{b}{8}$ and $a + b = 3060$.
The first equation is equivalent to $b = \frac{4}{5}a$, and substituting this into the second equation gives $a + \frac{4}{5}a = 3060$.
This means that $\frac{9}{5}a = 3060$, so $a = 1700$.

\section{Cowculus}
If the workbook's weight is $x$ pounds, then the textbook's weight is $0.8x$.
Therefore, $x = 0.8x + 10$, so $0.2x = 10$ and $x = 50$.
The workbook weights $50$ pounds, so the textbook weights $40$ pounds and the combined weight is $90$ pounds.

\section{Pocowmon Cards}
Let $a$ be the number of cards that Bessie has and $b$ be the number of cards that Bailey has.
From what Bessie said, we have $\frac{a + 4}{3} = \frac{b - 4}{2}$.
From what Bailey said, we have $b + 18 = 4(a - 18)$.
We can simplify the first equation to $2a - 3b = -20$ and the second equation to $4a - b = 90$.
Multiplying the first equation by $2$ and the subtracting the two equations gives $-5b = -110$, so $b = 22$.
Substituting this into either of the two equations lets us find $a$, which is $28$.

\section{Spot the Brown Cows}
We can use a two-way table to represent the data:
\begin{center}
	\begin{tabular}{|c|c|c|c|}
		\hline
                 & Brown spots & No brown spots & Total \\ \hline
		Calf     & $15$        &                & $41$  \\ \hline
		Not calf &             & $45$           &       \\ \hline
		Total    &             &                & $100$ \\ \hline
	\end{tabular}
\end{center}
For any row or column with two known numbers, we can fill in the remaining empty space by adding or subtracting.
The number of calves with no brown spots is $41 - 15 = 26$, so the total number of cows with no brown spots is $26 + 45 = 71$.
The number of cows with brown spots is then $100 - 71 = 29$.
The completed two-way table looks like this:
\begin{center}
	\begin{tabular}{|c|c|c|c|}
		\hline
                 & Brown spots & No brown spots & Total \\ \hline
		Calf     & $15$        & $26$           & $41$  \\ \hline
		Not calf & $14$        & $45$           & $59$  \\ \hline
		Total    & $29$        & $71$           & $100$ \\ \hline
	\end{tabular}
\end{center}

\section{Look Mom No Proof!}
The first rule in the paper checks for divisibility by $2$, the second rule checks for divisibility by $5$, and the third rule checks for divisibility by $3$.
Therefore, the paper is claiming that any integer greater than $5$ which is not a multiple of $2$, $3$, or $5$ is either a prime or a power of a prime.
To disprove this, we can find a number that is not a multiple of $2$, $3$, or $5$ and is not a prime or a power of a prime.
Our number must have at least two different prime factors in order for it to not be a power of a prime, and we can't have $2$, $3$, or $5$ as prime factors.
We can just take any two primes that aren't $2$, $3$, or $5$ and multiply them together.
For example, one answer would be $7 \cdot 11 = 77$.

\section{Holy Cow, A New Pen!}
The area of a triangle with base $b$ and height $h$ is $\frac{bh}{2}$.
The triangles $ABD$ and $ACD$ have the same height, so let $h$ represent this height.
The area of the left triangle is $\frac{12h}{2} = 6h$ and the area of the right triangle is $\frac{7h}{2}$.
The ratio between these two is $\frac{6h}{\frac{7h}{2}} = \frac{6}{\frac{7}{2}} = \frac{12}{7}$, or $12 : 7$.
This is the same as the ratio between the base lengths of the triangles, since a triangle's area is proportional to its base length when the height is constant.

\section{Moodern Art}
We can find the dimensions of the wall by adding the dimensions of the painting with the distances from the side of the painting to the edges of the wall.
The area of the wall is $(1829 + 94 + 182)(173 + 87 + 29) = 608345$ square centimeters.
We can then subtract the area of the painting to get $608345 - 94 \cdot 87 = 600167$ square centimeters as the area of the part of the wall that is not covered by the painting.

\section{Revomootion}
To maximize the number of cows in one pasture, we should minimize the number of cows in the other pastures.
One pasture can have $2$ cows.
The pasture with the next smallest number of cows would have $4$ cows, since $2$ is already used and we can't have exactly $3$ cows.
Next we can have pastures with $6$ cows, $7$ cows, and $8$ cows.
If we subtract $2 + 4 + 6 + 7 + 8$ from $42$, we get $15$ as the number of cows in the remaining pasture.

\section{On the Moove}
After $2$ hours, Bessie is $42 \cdot 2 = 84$ miles north of the barn.
Bailey is $84 - 30 = 54$ miles north of the barn.
Bailey traveled $54$ miels north in $2$ hours, so she was travelling at $\frac{54}{2} = 27$ miles per hour to the north.

\section{Cowld}
First, divide the pattern in to a grid of nine squares.
We can rearrange the four black triangles in the pattern to form two squares, so in total three out of nine squares in the pattern are black.
The area of the pattern is $90 \cdot 90 = 8100$ square inches and a third of it is black, so the area of the black parts in the pattern is $\frac{8100}{3} = 2700$ square inches.
The pattern is repeated $36$ times, so the total area of the black fabric is $2700 \cdot 36 = 97200$ square inches.

\section{Cownt the Rectangles}
\end{document}
